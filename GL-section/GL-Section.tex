GrainLearning is a calibration toolbox developed by Cheng et al., utilizes the recursive Bayesian algorithm to estimate the uncertainty parameters in DPM. Initially, a wide range of parameter space is quasi-randomly sampled numbers from the initial guess range to create a prior distribution of each parameter. Then, conditioned on the experimental values, the posterior distribution of the parameters is updated recursively by Sequential Monte-Carlo Filtering (SMC Filter) and fitted to a Gaussian Mixture Model. This process is done iteratively, until a desired value that minimises the loss function is reach, typically 3 iterations. 

\begin{algorithm}
    \caption{GrainLearning}\label{algorithm:GrainLearning}
    \begin{algorithmic}[1]
    \Procedure{Calibration}{}
    \State$\textit{input} \gets \text{measurement \textbf{y}, DEM solver \(\textbf{x} = F(\theta))}$
    \State $i \gets \textit{patlen}$
    \BState \emph{top}:
    \If {$i > \textit{stringlen}$} \Return false
    \EndIf
    \State $j \gets \textit{patlen}$
    \BState \emph{loop}:
    \If {$\textit{string}(i) = \textit{path}(j)$}
    \State $j \gets j-1$.
    \State $i \gets i-1$.
    \State \textbf{goto} \emph{loop}.
    \State \textbf{close};
    \EndIf
    \State $i \gets i+\max(\textit{delta}_1(\textit{string}(i)),\textit{delta}_2(j))$.
    \State \textbf{goto} \emph{top}.
    \EndProcedure
    \end{algorithmic}
    \end{algorithm}
    