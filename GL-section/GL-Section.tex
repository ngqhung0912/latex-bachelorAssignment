GrainLearning is a calibration toolbox developed by Cheng et al., utilizes the recursive Bayesian algorithm to estimate the uncertainty parameters in DPM. Initially, a wide range of parameter space is quasi-randomly sampled numbers from the initial guess range to create a prior distribution of each parameter. Then, conditioned on the experimental values, the posterior distribution of the parameters is updated recursively by Sequential Monte-Carlo Filtering (SMC Filter) and fitted to a Gaussian Mixture Model. This process is done iteratively, until a desired value that minimises the loss function is reach, typically 3 iterations. 

\begin{algorithm}
    \caption{GrainLearning}\label{algorithm:GrainLearning}
    \begin{algorithmic}
        \State$\textbf{Input:}$
        \State$\hspace*{5mm}\text{\textbf{y}}$ \Comment{\text{Experimental values}}
        \State$\hspace*{5mm} \textbf{x = F$(\Theta)$} \gets \text{DEM solver}$
        \State$\hspace*{5mm}\text{Initial guess range }(\Theta_{min}, \Theta_{max}) $
        \Function{Calibration}{$\textbf{Input}$}
            \For{k = 0 \textbf{to} K - 1}
                \IIf {$k = 0$} $\text{sample quasi-random prior distribution } \Theta_{k}^{(i)} \sim p_0(\Theta)$ 
                \IIf {$k > 0$} $\Theta_{k}^{(i)} \sim p_{k-1}(\Theta \mid y_{1:T})$
                \State$\text{Assume the normalized covariance parameter \sigma}$

            \EndFor
        \EndFunction
    \end{algorithmic}
\end{algorithm}
    