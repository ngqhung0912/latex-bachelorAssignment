
\begin{figure}[H]
    \centering
    \includegraphics[scale=0.5]{granularExample.png}
    \caption{Examples of Granular Materials.\cite{granularExample}}
    \label{fig:granularExample}
\end{figure}

Granular material is a family of material characterized by its large bulk of densely packed particles, ranging from nanometers to centimeters \cite{introGranular2}, and is able to resist deformation and form heaps, i.e., behave like a solid and withstand strong shear force \cite{introGranular3}. Simple examples of granular materials include sand, gravel, clays, seeds, nuts, and all ranges of powders such as coffee powder, cement powder, which is shown in figure \ref{fig:granularExample}. Furthermore, many processes and equipments in chemical plants use granular materials, such as catalysis, adsorption, and heat exchangers. Granular materials are projected to make about half of the products and three-quarters of the raw materials used in the chemical industry \cite{introGranular}. Thus, understanding how granular materials behave is of great significance. 

The simulation of granular material's bulk mechanical behavior is done using Discrete Particle Model (DPM, or Discrete Element Method - DEM), which generates the movement of individual particles to capture the macro-scale behavior. The DPM is a family of numerical methods for computing the motion of a large number of particles \cite{Weng:2015}, first proposed by Cundall and Strack in the 1970s \cite{cundallstrack}.
Since the properties of granular materials differ wildly, these simulations require an extensive calibration process designed individually for each type of granular material. Some parameters of the granular material model can be measured directly, such as size distribution or density. However, other parameters are effective parameters (i.e., they result from a simplified particle model) and thus cannot be directly measured. These parameters are then calibrated by choosing a few standard calibration setups (rotating drum, heap test, ring shear cell) and simulating these setups in a DPM simulation, and the missing parameters are determined such that the response of the experimental and simulation setups match.

Recently, coupled with the raise of Machine Learning in other fileds, it has also been applied to solve the calibration problem. This has been done using a Neural Network \cite{nn-calibration, NN-GA, NN-coarse}, Genetic Algorithm \cite{ga-calibration}, and a recursive Bayesian sequential Monte-Carlo filtering algorithm named GrainLearning \cite{grainLearning}. In this Assignment, two Machine Learning algorithms will be discussed: Neural Network and GrainLearning. 

%todo is this true? 

These two algorithms set to treat the calibration problem in two different ways, and likewise, solve it in two different ways: While GrainLearning looks to identify the microparameters from the experimental and DEM simulations's bulk parameters (inverse problem), the Neural Network will help generating a database that can map different microparameters combinations to their corresponding bulk parameters, generated by DEM simulations. In other word, NN will learn the built-in relationship between the micro- and macroparameters of the Discrete Particle Model, thus allow a much faster prediction compare to a full DEM simulation. 
One advantage of GrainLearning compares to other Machine Learning algorithms such as Neural Network is that it is an unsupervised learning algorithm, i.e., it can starts calibrating with a minimal amount of input information. A normal calibration routine, currently implemented in MercuryDPM would only need the measurements data, parameter range, and the importance weight of each measurement (depends on the modeller's knowledge). Meanwhile, the current approach mentioned in \cite{nn-calibration, NN-GA, NN-coarse} would requires modeller to define a different NN model for each material, train it using a set of DEM simulations, and then validate the correct combinations of input-output by experiment. And although this process can be automated, to date the author has not been aware of any study implemented a fully-automated calibration routine using Neural Network. 


 \textbf{todo here: A paragraph describes each section}


