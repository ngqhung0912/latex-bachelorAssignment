Granular material is a family of materials characterized by its enormous bulk of densely packed particles, ranging from nanometers to centimeters~\cite{introGranular2}, and is able to resist deformation and form heaps, i.e., behave like a solid and withstand strong shear force~\cite{introGranular3}. Examples of granular materials include sand, gravel, clays, seeds, nuts, and all ranges of powders such as coffee powder and cement powder, shown in figure~\ref{fig:granularExample}. Furthermore, many processes and equipment in chemical plants use granular materials, such as catalysis, adsorption, and heat exchangers. Granular materials are projected to make about half of the products and three-quarters of the raw materials used in the chemical industry~\cite{introGranular}. Thus, understanding how granular materials behave is of great significance. 

\begin{figure}[H]
 \centering
 \includegraphics[scale=0.5]{granularExample.png}
 \caption{Examples of Granular Materials~\cite{granularExample}.}
 \label{fig:granularExample}
\end{figure}

The simulation of granular material's bulk mechanical behavior is done using the Discrete Particle Model (DPM, or Discrete Element Method~-~DEM), which generates the movement of individual particles to capture the macro-scale behavior. The DPM is a family of numerical methods for computing the motion of a large number of particles~\cite{Weng:2015}, first proposed by Cundall and Strack in the 1970s~\cite{cundallstrack}.
Since the properties of granular materials differ wildly, these simulations require an extensive calibration process designed individually for each type of granular material. Some parameters of the granular material model can be measured directly, such as size distribution or density. However, other parameters are effective parameters (i.e., they result from a particle-particle contact) and thus cannot be directly measured. These parameters are then calibrated by choosing a few standard calibration setups (rotating drum, heap test, ring shear cell) and simulating these setups in a DPM simulation. The missing parameters are determined such that the response of the experimental and simulation setups match.

Recently, coupled with the rise of Machine Learning in other fields, it has also been applied to solve the calibration problem. This has been done using a Neural Network~\cite{nn-calibration, NN-GA, NN-coarse, YE2019292}, Genetic Algorithm~\cite{ga-calibration}, and a recursive Bayesian sequential Monte-Carlo filtering algorithm named GrainLearning~\cite{grainLearning}. This research will discuss three Machine Learning algorithms: Neural Network, Random Forest (supervised model), and GrainLearning (unsupervised model). These three algorithms are set to treat the calibration problem in two different ways: While GrainLearning looks to identify the microparameters from the experimental and DEM simulations' bulk parameters (inverse problem), the supervised models will help generate a database that can map different microparameters combinations to their corresponding bulk parameters, generated by DEM simulations. In other words, NN and RF will learn the built-in relationship between the micro- and macroparameters of the Discrete Particle Model, thus allowing a much faster prediction compare to a full DEM simulation. One advantage of GrainLearning compared to other Machine Learning algorithms such as Neural Network is that it is an unsupervised learning algorithm, i.e., it can start calibrating with the minimal input information. However, each material needs calibration for multiple bulk parameters, i.e., Static Angle of Repose, Dynamic Angle of Repose, shear tester, etc.~since a set of microparameters that are valid for one bulk parameter might not be valid for another~\cite{reviewCalibration}. Therefore, scaling up with a Neural Network model might be simpler since a Neural Network can produce multiple valid combinations for each bulk parameter. 

In the next two sections, the characterization and experimental method will be discussed, including static Angle of Repose, Discrete Particle Model, contact laws, and the material used in the simulation. Section~\ref{section:Calibration} will discuss in detail the different approaches and methods of each model. The result of GrainLearning will be discussed in section~\ref{section:GLPerformance}, while section~\ref{section:supervisedPerformance} discusses the supervised model's performance. Subsequently, section~\ref{section:discussion} will compare the models, discuss the strength and weaknesses of each model, and the limitations. Finally, section~\ref{section:conclusion} will summarize and conclude the research with recommendations to further expand the use of machine learning in calibration of DPM.

