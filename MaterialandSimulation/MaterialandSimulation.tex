Experimental data on quartz sand is provided by Derakhshani et al.~\cite{DERAKHSHANI2015127}. The density of quartz sand is $\rho = 2653~kg/m^3$, and the particle size distribution (PSD) given in Table~\ref{table:psd}, with the static AoR of $33^{\circ}$. Meanwhile, experimental data on limestone is provided by Shi et al.~\cite{SHI2020183}, specifically the Eskal 150 limestone, since this material has a similar experimental static AoR and density, while the PSD is in a much lower range. The density of Eskal 150 is $2761~kg/m^3$, and the static AoR is also $33^{\circ}$.


\begin{table}[H]
    \centering
    \begin{tabular}{c|cc}
    Material & Diameter (µm) & Cumulative volume distribution (\%) \\ \hline
    Limestone & 97 & 10 \\
     & 138 & 50 \\
     & 194 & 90 \\ \hline
    Quartz sand & 300 & 6.21 \\
     & 425 & 24.50 \\
     & 500 & 50.55 \\
     & 600 & 100
    \end{tabular}
    \caption{Particle Size Distribution of materials.}\label{table:psd}
\end{table}
        

The experimental data described above will be used as constant input values for each DEM simulation. In addition, four variables will be tested to determine their respective static AoR:\@
\begin{itemize}
    \item Restitution coefficient: Ratio between the velocity of the particle before and after collision. The restitution coefficient is in the range of 0 to 1, with 1 denotes a perfectly elastic collision.
    \item Sliding friction: Force that acts on the opposite direction of the movement between two particles when they collide. 
    \item Rolling friction: Force resists the rolling motion of the particle. 
    \item Bond number: Ratio between gravitational force that act on the particle and the surface tension force, i.e., the easiness of movement due to gravity. 
\end{itemize}

In addition to the range bound of Restitution Coefficient, all other microparameters has a positive bound~-~the value ranges from 0 to $\infty$, however in most realistic case the value is also smaller than 1.



