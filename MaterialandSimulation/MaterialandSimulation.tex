Experimental data on quartz sand is provided by Derakhshani et al.~\cite{DERAKHSHANI2015127}. The density of quartz sand is $\rho = 2653~kg/m^3$, and the particle size distribution (PSD) given in Table~\ref{table:psd}, with the static AoR of $33^{\circ}$. Meanwhile, experimental data on limestone is provided by Shi et al.~\cite{SHI2020183}, specifically the Eskal 150 limestone, since this material has a similar static AoR and density ($33^{\circ}$ and $2761~kg/m^3$), while the PSD is in a much lower range. 

The experimental data described above will be used as constant input values for each DEM simulation, among with collision time $ = 0.068$ms. In addition, four variables~-~including three described in eq.~\ref{eq:sliding},~\ref{eq:rolling}, and~\ref{eq:bond} will be varied to determine their respective static AoR:~Restitution coefficient, sliding friction, rolling friction, and bond number. The restitution coefficient is the ratio between the particle's velocity before and after the collision. The parameter range of restitution coefficient is 0 to 1, with 1 denoting a perfectly elastic collision. All other microparameters has a positive bound~-~the value ranging from 0 to $\infty$; however, in most realistic case, the value is also smaller than 1. 

\begin{table}[H]
    \centering
    \begin{tabular}{c|cc}
    Material & Diameter (µm) & Cumulative volume distribution (\%) \\ \hline
    Limestone & 97 & 10 \\
     & 138 & 50 \\
     & 194 & 90 \\ \hline
    Quartz sand & 300 & 6.21 \\
     & 425 & 24.50 \\
     & 500 & 50.55 \\
     & 600 & 100
    \end{tabular}
    \caption{Particle Size Distribution of materials.}\label{table:psd}
\end{table}
        



