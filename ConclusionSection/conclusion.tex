This research has demonstrated three methods of using machine learning to takle the calibration problem of Discrete Particle Model, namely GrainLearning, Neural Network, and Random Forest algorithm within the scope of one bulk parameter, the static angle of repose. Overall, it has been found out that all algorithms have the ability to search for a correct microparameters combinations, to reproduce the exact static AoR in MercuryDPM compare to the experimental value~-~albeit in vastly different ways. While GL iteratively samples the new set of parameter that is getting closer to the optimal value conditioned on previously-learned knowledge, NN and RF models require a two-step method: training with different DEM simulations, then feed it with multiple combinations and select the output that match the experimental values, and verify it using a DEM simulation. It has also shown that GrainLearning has an inconsistent performance 