This research has demonstrated three methods of using machine learning to tackle the calibration problem of the Discrete Particle Model, namely GrainLearning, Neural Network, and Random Forest algorithm within the scope of one bulk parameter, the static angle of repose. Overall, it has been found that all tested algorithms can search for correct microparameters combinations to reproduce the exact static AoR in MercuryDPM compared to the experimental value~-~albeit in vastly different ways. While GL iteratively samples the new set of parameters that is getting closer to the optimal value conditioned on previously-learned knowledge, NN and RF models require a three-step method: training with different DEM simulations, then feeding it with multiple combinations, and selecting the output that matches the experimental values, and verify it using a DEM simulation. It has also shown that GrainLearning has an inconsistent performance, presumably due to the guessing range provided to the algorithm being too large. Meanwhile, the NN and especially the RF model demonstrate the capability to learn the mapping between the DEM input parameters and its bulk behavior for the given linear spring-dashpot contact law. 

Future research could focus on examining the ability of GrainLearning on the static angle of repose and expand the study on the Random Forest algorithm on different bulk parameters~-~and on different contact laws. 

\section{Acknowledgements}

I would like to express my gratitude toward dr. Thomas Weinhart, for the constant support I received before and during the project, for the time you spent answering every question I had in mind, and for the helpful feedback on my first draft introduction and draft report. Additionally, I am grateful to Prof. dr. Anthony Thornton for the ideas and explanations I received in the weekly meeting. 

Besides my supervisor, I would like to thank Martin Wilens and Stefano Onofri for the immense help when I was trying to set up the DEM simulations in the HPC using slurm. I thank dr. Hanneke Becht and Shane Cordell for the helpful comments regarding my writing style. And thank to my peers for the feedback during my proposal presentations before the project. 
Finally, I would be remiss in not mentioning my study advisor from the CSE program, Nienke Oesterholt. Nienke has always been there for me whenever I have doubts that are both study- and non-study related. 
