\documentclass{article}

\usepackage{xr-hyper}
\usepackage{subfiles}
\usepackage{amsmath}
\usepackage[english]{babel}
\usepackage{graphicx}
\usepackage{float}
\usepackage[letterpaper,top=2cm,bottom=2cm,left=3cm,right=3cm,marginparwidth=1.75cm]{geometry}
\usepackage{parskip}
\usepackage{amsmath}
\usepackage{graphicx}
\usepackage[colorlinks=true, allcolors=black]{hyperref}
\usepackage{caption}
\usepackage{algorithm}
\usepackage[noend]{algpseudocode}
\usepackage{nicefrac}
\usepackage{bookmark}
\usepackage{titlesec}
\usepackage{tabularx}

\makeatletter
\def\BState{\State\hskip-\ALG@thistlm}
\makeatother

\algnewcommand{\IIf}[1]{\State\algorithmicif\ #1\ \algorithmicthen}
\algnewcommand{\EndIIf}{\unskip\ \algorithmicend\ \algorithmicif}
\algnewcommand{\LineComment}[1]{\State \(\triangleright\) #1}

\setcounter{secnumdepth}{4}
\titleformat{\paragraph}
{\normalfont\normalsize\itshape}{\theparagraph}{1em}{}
\titlespacing*{\paragraph}
{0pt}{3.25ex plus 1ex minus .2ex}{1.5ex plus .2ex}
\renewcommand{\arraystretch}{1.5}

% elsarticle-num
\bibliographystyle{elsarticle-num}
\graphicspath{ {./Images/} }
\captionsetup[figure]{font=small}
\title{Machine Learning in the calibration process of Discrete Particle Model: The case with Angle of Repose}
\author{Quang Hung Nguyen\\[1ex] \small Head Supervisor: Thomas Weinhart, Daily Supervisor: Anthony Thornton, Additional member: Chen Kuan. \\
\small Multi-Scale Mechanics Group, Faculty of Engineering Technology, University of Twente.} 
\date{}

\begin{document}
\maketitle

\begin{center}
    \Large\textbf{Abstract}
\end{center}


\begin{center}
    This is where, stuff, happens. Literally. But not yet. Please wait. 
\end{center}



\pagebreak

\tableofcontents

\pagebreak

\section{Introduction}\label{section:Introduction}
\documentclass[../BachelorAssignment.tex]{subfiles}
    

\begin{document}
\graphicspath{{\subfix{../Images/}}}

\begin{figure}[H]
    \centering
    \includegraphics[scale=0.5]{granularExample.png}
    \caption{Examples of Granular Materials.\cite{granularExample}}
    \label{fig:granularExample}
\end{figure}


 Granular material is a family of material characterized by its large bulk of densely packed particles, ranging from nanometers to centimeters \cite{introGranular2}, and is able to resist deformation and form heaps, i.e., behave like a solid and withstand strong shear force \cite{introGranular3}. Simple examples of granular materials include sand, gravel, clays, seeds, nuts, and all ranges of powders such as coffee powder, cement powder, which is shown in figure \ref{fig:granularExample}. Furthermore, many processes and equipments in chemical plants use granular materials, such as catalysis, adsorption, and heat exchangers. Granular materials are projected to make about half of the products and three-quarters of the raw materials used in the chemical industry \cite{introGranular}. Thus, understanding how granular materials behave is of great significance. 


Granular material's bulk mechanical behavior is simulated using a Discrete Particle Model (DPM, or Discrete Element Method - DEM), which generates the movement of individual particles to capture the macro-scale behavior. The DPM is a family of numerical methods for computing the motion of a large number of particles \cite{Weng:2015}. Since the properties of granular materials differ wildly, these simulations require an extensive calibration process designed individually for each type of granular material. Some parameters of the granular material model can be measured directly, such as size distribution or density. However, other parameters are effective parameters (i.e., they result from a simplified particle model) and thus cannot be directly measured. These parameters are then calibrated by choosing a few standard calibration setups (rotating drum, heap test, ring shear cell) and simulating these setups in a DPM simulation, and the missing parameters are determined such that the response of the experimental and simulation setups match. This has been done using a Convolutional Neural Network \cite{nn-calibration} and Genetic Algorithm \cite{ga-calibration}. However, these methods are very costly in terms of  time and resources, since it requires a large number of DEM simulations. In this research assignment, another algorithm will be studied: GrainLearning - a Python-based Bayesian calibration tool developed by H.Cheng et al. \cite{grainLearning}. GrainLearning utilizes the recursive Bayesian algorithm to estimate the uncertainty parameters in DPM. As mentioned above, each model requires extensive calibration, and using GrainLearning can make the calibration process more effective by iteratively searching for the optimized parameters, effectively reducing the number of DEM simulations required.

GrainLearning has been implemented in MercuryDPM and produces satisfactory results in many cases. However, it is far from clear whether the chosen optimization algorithm is optimal, i.e., whether the calibration results in an optimal parameter set and whether the optimum could be reached faster since currently, a calibration process could run for hours. Therefore, the research question here is that: Is there a more efficient way to calibrate the parameters for DEM simulation using GrainLearning?

\end{document}


% \section{Timeline and procedure}

% \subfile{TimelineSections/Timeline.tex}

% \section{Rebuttal}

% \subfile{RebuttalSections/Rebuttal.tex}


\section{Characterisation of granular materials}


There are no established standard of characterisation measurements for granular materials. Common measurements include heap test, rotating drum test, linear/ring shear cell test, and the silo flow test,\ldots, in which the output is the bulk parameter, which defines how the granular material behaves in large quantity~-~such as angle of repose (AoR), shear stress, flow rate, etc. 

This research is focused on one of the most important bulk parameters to describe the characteristics of granular materials~-~the static angle of repose. Static AoR, which is described in Fig.~\ref{fig:StaticAoR}, is defined as the angle that granular solids forms when it piled with a flat surface, and is essential to characterise the coarseness and smootheness of materials. This in turn can help designing a process involved with the material~-~lower static AoR implies more flowable and thus easier to transport with less energy~\cite{TEFERRA201945}. 


\begin{figure}[t]
    \centering
    \includegraphics[scale=0.5]{StaticAoR.png}
    \caption{Static Angle of Repose measurement steps~\cite{Rackl:2018te}.}\label{fig:StaticAoR}
\end{figure}





\section{Simulation Method}\label{section:ExpMethod}


The DEM package used in this Assignment is MercuryDPM. MercuryDPM is an open-source DEM software package, developed by Thomas et al.\cite{MercuryDPM}. 






\subsection{Material and simulation properties}\label{section:matprop}

Experimental data on quartz sand is provided by Derakhshani et al.~\cite{DERAKHSHANI2015127}. The density of quartz sand is $\rho = 2653~kg/m^3$, and the particle size distribution (PSD) given in Table~\ref{table:psd}, with the static AoR of $33^{\circ}$. Meanwhile, experimental data on limestone is provided by Shi et al.~\cite{SHI2020183}, specifically the Eskal 150 limestone, since this material has a similar static AoR and density ($33^{\circ}$ and $2761~kg/m^3$), while the PSD is in a much lower range. 


\begin{table}[H]
    \centering
    \begin{tabular}{c|cc}
    Material & Diameter (µm) & Cumulative volume distribution (\%) \\ \hline
    Limestone & 97 & 10 \\
     & 138 & 50 \\
     & 194 & 90 \\ \hline
    Quartz sand & 300 & 6.21 \\
     & 425 & 24.50 \\
     & 500 & 50.55 \\
     & 600 & 100
    \end{tabular}
    \caption{Particle Size Distribution of materials.}\label{table:psd}
\end{table}
        

The experimental data described above will be used as constant input values for each DEM simulation. In addition, four 
variables~-~including three described in eq.~\ref{eq:sliding},~\ref{eq:rolling}, and~\ref{eq:bond} will be tested to determine their respective static AoR:~Restitution coefficient, sliding friction, rolling friction, and bond number. The restitution coefficient is the ratio between the particle's velocity before and after the collision. The parameter range of restitution coefficient is 0 to 1, with 1 denoting a perfectly elastic collision. All other microparameters has a positive bound~-~the value ranging from 0 to $\infty$; however, in most realistic case, the value is also smaller than 1. 




\section{Calibration of Discrete Particle Model}\label{section:Calibration}
    
\subsection{GrainLearning}\label{section:GLtheory}
GrainLearning is a calibration toolbox developed by Cheng et al., utilizes the recursive Bayesian algorithm to estimate the uncertainty parameters in DPM. Initially, a wide range of parameter space is quasi-randomly sampled numbers from the initial guess range to create a prior distribution of each parameter. Then, conditioned on the experimental values, the posterior distribution of the parameters is updated recursively by Sequential Monte-Carlo Filtering (SMC Filter) and fitted to a Gaussian Mixture Model. This process is done iteratively, until a desired value that minimises the loss function is reach, typically 3 iterations. 

\begin{algorithm}
    \caption{GrainLearning}\label{algorithm:GrainLearning}
    \begin{algorithmic}
        \State$\textbf{Input:}$
        \State$\hspace*{5mm}\text{\textbf{y}}$ \Comment{\text{Experimental values}}
        \State$\hspace*{5mm} \textbf{x = F$(\Theta)$} \gets \text{DEM solver}$
        \State$\hspace*{5mm}\text{Initial guess range }(\Theta_{min}, \Theta_{max}) $
        \Function{Calibration}{$\textbf{Input}$}
            \For{k = 0 \textbf{to} K - 1}
                \IIf {$k = 0$} $\text{sample quasi-random prior distribution } \Theta_{k}^{(i)} \sim p_0(\Theta)$ 
                \IIf {$k > 0$} $\Theta_{k}^{(i)} \sim p_{k-1}(\Theta \mid y_{1:T})$
                \State$\text{Assume the normalized covariance parameter \sigma}$

            \EndFor
        \EndFunction
    \end{algorithmic}
\end{algorithm}
    


\subsection{Neural Network}
Artificial Neural Network (ANN) is a set of algorithms that seeks to identify correlations in data utilizing a technique that inspired by the way human brain operates - mimicking how each neurons in the brain signaling each other. The most basic ANN models is the Feed-forward Multilayer Perceptron Neural Network (MLPNN), in which the purpose is to define the mapping between the input and output \(y = f(x;\theta)\), and approximate the parameter \(\theta\) which results in the best possible function. In MLPNN, the data will flows in one direction from the input to output, hence the name feed-forward. Like other supervised learning algorithms, a MLPNN need to be trained before it  accurately describe the relations between the input and output. This is typically done by feeding the network with pre-labeled data, compare the model's output with the desired output, and update the weights parameter \(\theta\) - a process called backpropagation. In the context of this Assignment, Neural Network (NN) will be used when referring to Feedforward Multilayer Perceptron Neural Network. 



\subsubsection{Designing a Neural Network}
In order to choose a correct NN model that can describes the relationship of micro- and macroparameters as indicated in the contact law, a "bottom-up" method is employed.

Initially, a simple one-input one-output NN will be built, to assess how many layers (and neurons) it would take to learn a quadratic relationship \(y = x^2\), and inverse relationship \(y = 1/x\).

In the next step, a slightly more complex problem is analysed: a 4-D input and 4-D output model, with contact laws described in table xxx. Finally, a fully-functional model will be coupled with DEM simulations from MercuryDPM.   

Certain reccomendations will also be taken into account, (7 and 15? )




\subsection{Random Forest algorithm}

This section will discuss different concepts of a Random Forest (RF) algorithm, starting with the basis of the RF:\@ decision tree. Decision tree is an algorithm that generates a tree graph of decisions based on the input provided and their possible outcomes, and as a consequence, it partitions the input space into multiple regions, with each region accounting for a different outcome~\cite{murphyML}. An example of a simple decision tree based on two inputs is shown in figure~\ref{fig:decisionTree}. 

\begin{figure}[H]
    \centering
    \includegraphics[scale=0.8]{regressionTree.png}
    \includegraphics[scale=0.8]{decisionTree.png}
    \caption{Example of a decision tree regressor on a two-input problem.~\cite{murphyML}}\label{fig:decisionTree}
\end{figure}

The most significant advantage of the decision tree, and subsequently, random forest algorithm, is that it is relatively simple, explainable, easy to train and interpolate with little computational resources. However, one crucial drawback of a decision tree is its instability: minor data changes might affect the tree structure, making the decision tree a high variance estimators~\cite{murphyML}. Attempts have been made to reduce the uncertainty of the decision tree, one of which is the so-called Random Forest algorithm, which Breiman proposed in 2001~\cite{BreimanRF}. RF made up for the high variance of a single decision tree by averaging the results over a ``forest'' of decision trees, with each tree represents an independent sampled vectors. The concept of decision tree and RF, therefore, fit within the scope of the calibration problem.


\subsection{Training and evaluation method for supervised models}  

To train the NN and RF model, 500 DEM simulations with randomized combinations of input parameters have been performed, in order to create a database which maps DEM microparameters to static AoR. Other 125 simulations did not finish on the time constraint set, and as a results marked as an inaccurate combination. 




\input{MaterialandSimulation/CalibrationMethod.tex}

% section result and analysis 

\documentclass[../BachelorAssignment.tex]{subfiles}
    

\begin{document}
\graphicspath{{\subfix{../Images/}}}


\subsection{}

\end{document}

\section{Conclusion} 
This research has demonstrated three methods of using machine learning to takle the calibration problem of Discrete Particle Model, namely GrainLearning, Neural Network, and Random Forest algorithm within the scope of one bulk parameter, the static angle of repose. Overall, it has been found out that all algorithms have the ability to search for a correct microparameters combinations, to reproduce the exact static AoR in MercuryDPM compare to the experimental value~-~albeit in vastly different ways. While GL iteratively samples the new set of parameter that is getting closer to the optimal value conditioned on previously-learned knowledge, NN and RF models require a two-step method: training with different DEM simulations, then feed it with multiple combinations and select the output that match the experimental values, and verify it using a DEM simulation. It has also shown that GrainLearning has an inconsistent performance 

\pagebreak 



\bibliography{citation}

\begin{center}
© 2022. This work is licensed under a CC BY 4.0 license. 
\end{center}


\end{document}
 