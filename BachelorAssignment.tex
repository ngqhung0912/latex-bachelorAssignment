\documentclass{article}

\usepackage{subfiles}
\usepackage{amsmath}
\usepackage[english]{babel}
\usepackage{graphicx}
\usepackage{float}
\usepackage[letterpaper,top=2cm,bottom=2cm,left=3cm,right=3cm,marginparwidth=1.75cm]{geometry}
\usepackage{parskip}
\usepackage{amsmath}
\usepackage{graphicx}
\usepackage[colorlinks=true, allcolors=black]{hyperref}
\usepackage{caption}

\bibliographystyle{elsarticle-num}
\newcommand*{\subfilesbibliography}[1]{%
 \expandafter\ifx\csname ver@subfiles.cls\endcsname\relax
   \expandafter\@secondoftwo
 \else
   \expandafter\@firstoftwo
  \fi
  {\bibliography{#1}}
  {}%
}
\graphicspath{ {./Images/} }
\captionsetup[figure]{font=small}
\title{Machine Learning in the calibration process of MercuryDPM Discrete Particle Model}
\author{Quang Hung Nguyen\\[1ex] \small Head Supervisor: Thomas Weinhart, Daily Supervisor: Anthony Thornton, Additional member: Chen Kuan. \\
\small Multi-Scale Mechanics Group, Faculty of Engineering Technology, University of Twente.} 
\date{}

\begin{document}
\maketitle

\section{Introduction}


\subfile{IntroSections/Intro.tex}

% \section{Timeline and procedure}

% \subfile{TimelineSections/Timeline.tex}

% \section{Rebuttal}

% \subfile{RebuttalSections/Rebuttal.tex}

% \pagebreak


\section{Calibration}
\subsection{DEM Simulations}
\subfile{MercurySection/MercuryandSimulation.tex}

\subsection{GrainLearning}


\subsection{Artificial Neural Network}
\subfile{ANNSection/ANN.tex}

\pagebreak  

\bibliography{citation}

\end{document}