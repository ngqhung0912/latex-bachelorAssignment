\documentclass{article}

% Language setting
% Replace `english' with e.g. `spanish' to change the document language
\usepackage[english]{babel}

% Set page size and margins
% Replace `letterpaper' with `a4paper' for UK/EU standard size
\usepackage[letterpaper,top=2cm,bottom=2cm,left=3cm,right=3cm,marginparwidth=1.75cm]{geometry}

% Useful packages
\usepackage{amsmath}
\usepackage{graphicx}
\usepackage[colorlinks=true, allcolors=blue]{hyperref}

\title{Using Machine Learning to optimize the calibration of Discrete Particle Model}
\author{Q.H. (Hung, Student B-CSE), student number s2096307}

\begin{document}
\maketitle
\begin {center}
\textbf{\LARGE Draft Introduction}
\end{center}

\section{Granular material}

 Granular material is a family of material characterized by its large bulk of densely packed particles, ranging from nanometers to centimeters, and is able to resist deformation and form heaps \cite{introGranular2, introGranular3}. Simple examples of granular materials include sand, gravel, clays, seeds, nuts, and all ranges of powders such as coffee powder, cement powder. Furthermore, many processes and equipments in chemical plants use granular materials, such as catalysis, adsorption, and heat exchangers. It is estimated that roughly one-half of the products and three-quarters of the raw materials in the chemical industry is in the form of granular materials \cite{introGranular}. Thus, understanding how granular materials behave is of great significance.

\section{Discrete Particle Model and calibration}

Granular material behavior is simulated using a Discrete Particle Model (DPM, or Discrete Element Method - DEM), which generates the movement of individual particles to capture the macro-scale behavior. As defined by X.Weng \cite{Weng:2015}, the DPM is a family of numerical methods for computing the motion of a large number of particles. Since the properties of granular materials differ wildly, these simulations require an extensive calibration process designed individually for each type of granular material. Some parameters of the granular material model can be measured directly, such as size distribution or density. However, other parameters are effective parameters (i.e., they result from a simplified particle model) and thus cannot be directly measured. These parameters are then calibrated by choosing a few standard calibration setups (rotating drum, heap test, ring shear cell) and simulating these setups in a DPM simulation, and the missing parameters are determined such that the response of the experimental and simulation setups match. 

GrainLearning is a Python-based Bayesian calibration tool, developed by H.Cheng et al. \cite{grainLearning}, which utilizes the recursive Bayesian algorithm to estimate the uncertainty parameters in DPM. As mentioned above, each model requires extensive calibration, and using GrainLearning can make the calibration process more effective by iteratively searching for the optimized parameters. 

\section{Goals of the Assignment}

This calibration technique has been implemented in MercuryDPM (6) and produces satisfactory results in many cases. However, it is far from clear whether the chosen optimization algorithm is optimal, i.e., whether the calibration results in an optimal parameter set and whether the optimum could be reached faster. Therefore, the first phase of the research is to study, analyse, and improve the calibration process of the GrainLearning algorithm. In the second phase, different Machine Learning models will be built and compared with GrainLearning. 

\bibliographystyle{ieeetr}
\bibliography{citation}

\end{document}