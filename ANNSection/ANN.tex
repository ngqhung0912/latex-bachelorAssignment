Artificial Neural Network (ANN) is a set of algorithms that seeks to identify correlations in data utilizing a technique inspired by how the human brain operates - mimicking how each neuron in the brain signals each other. The most basic ANN model is the Feed-forward Multilayer Perceptron Neural Network (MLPNN), in which the purpose is to define the mapping between the input and output \(y = f(x;\theta)\) and approximate the parameter \(\theta\) which results in the best possible function. In MLPNN, the data will flows in one direction from the input to the output, hence the name feed-forward. Like other supervised learning algorithms, an MLPNN needs to be trained before accurately describing the input and output relations. This is typically done by feeding the network with pre-labeled data, comparing the model's output with the desired output, and updating the weights parameter \(\theta\) - a process called backpropagation. In this assignment's context, Neural Network (NN) will be used when referring to Feedforward Multilayer Perceptron Neural Network, and NN models implemented in this research are provided by the open-source library \texttt{TensorFlow}~\cite{tensorflow2015-whitepaper}.

\subsubsection{Designing a Neural Network}

There are no general rules for determining the number of layers and the number of neurons per layer, and it depends heavily on each use case. While Benvenuti et al.~\cite{nn-calibration}, He et al.~\cite{NN-GA}, and Daniel et al.~\cite{NN-coarse} used only a single-layer ANN and varied the number of neurons, Ye et al.~\cite{YE2019292} vary both. However, the ultimate goal in both case is to find the combinations which result in the minimum error while also avoiding overfitting, i.e., the model excels on training but perform poorly on the validation step. In this case, for each simulation material, 250 models ranging from 2 to 15 layers and 5 to 15 neurons per layer are tested to determine the best model. Each model is trained for 50 epochs with a batch size of 32, and the metric used to grade the model is Mean Absolute Error. In addition, 20\% of the data will be saved for validation of the model. 

Another important component of a Neural Network is the activation function. Since each neuron performs calculation by multiplying the input with weight and adding a bias, the activation function's role would be introducing a non-linearity element into an otherwise linear neuron. According to Goodfellow et al.,~\cite{DL-Goodfellow}, Rectified Linear Unit (ReLU) is the recommendation for most Deep Learning models, with its ability to preserve much of the properties due to its near-linear shape. ReLU activation function is defined as $f(x) = max(0, x)$. 

\subsubsection{Training and evaluation method}  

To train the Neural Network model, 500 DEM simulations with randomized combinations of input parameters have been performed, in order to create a database which maps DEM microparameters to static AoR. Other 125 simulations \textbf{did not finish on the time constraint set, and as a results marked as an inaccurate combination.} Afterwards, over 1,000,000 different combinations, as describe in table~\ref{table:randomCombinations}, will be processed by the NN model. Any combinations that producess a static AoR within the $0.1\%$ margin of error to the experimental data is marked as a correct combination. This combination will then be evaluated independently by a DEM simulation, to verify the ability of NN model correctly describes bulk behavior of a material based on the given contact law. 

\begin{table}[H]
    \centering
    \begin{tabular}{c|cccc}
                     & Restitution Coefficient & Sliding Friction & Rolling Friction & Bond number  \\ \hline
    Range            & {[}0.5 1{]}             & {[}1e-5 1{]}     & {[}1e-5 1{]}     & {[}1e-5 1{]} \\
    Number of values & 30                      & 30               & 30               & 30          
    \end{tabular}
    \caption{Random evenly-spaced microparameters combinations}
    \label{table:randomCombinations}
\end{table}