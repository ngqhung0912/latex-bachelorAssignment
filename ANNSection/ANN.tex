Artificial Neural Network (ANN) is a set of algorithms that seeks to identify correlations in data utilizing a technique that inspired by the way human brain operates - mimicking how each neurons in the brain signaling each other. The most basic ANN models is the Feed-forward Multilayer Perceptron Neural Network (MLPNN), in which the purpose is to define the mapping between the input and output \(y = f(x;\theta)\), and approximate the parameter \(\theta\) which results in the best possible function. In MLPNN, the data will flows in one direction from the input to output, hence the name feed-forward. Like other supervised learning algorithms, a MLPNN need to be trained before it  accurately describe the relations between the input and output. This is typically done by feeding the network with pre-labeled data, compare the model's output with the desired output, and update the weights parameter \(\theta\) - a process called backpropagation. In the context of this Assignment, Neural Network (NN) will be used when referring to Feedforward Multilayer Perceptron Neural Network. 



\subsubsection{Designing a Neural Network}
In order to choose a correct NN model that can describes the relationship of micro- and macroparameters as indicated in the contact law, a "bottom-up" method is employed.

Initially, a simple one-input one-output NN will be built, to assess how many layers (and neurons) it would take to learn a quadratic relationship \(y = x^2\), and inverse relationship \(y = 1/x\).

In the next step, a slightly more complex problem is analysed: a 4-D input and 4-D output model, with contact laws described in table xxx. Finally, a fully-functional model will be coupled with DEM simulations from MercuryDPM.   

Certain reccomendations will also be taken into account, (7 and 15? )


