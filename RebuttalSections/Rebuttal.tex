\documentclass[../BachelorAssignment.tex]{subfiles}

\begin{document}

The most common feedback I received during the preparation phase was "unclear research questions". In the beginning, when my supervisors briefed the assignment to me, I intended to phase it ambiguously to have a broader area to define my Bachelor's Assignment while I was doing it. However, after discussing it again, we decided to redefine my research question to a smaller field, including the analysation and improvement of the GrainLearning algorithm in the calibration process of DEM simulations. The new research question is shown in the Goals of the Assignment section above. 
\par
During the first and second peer review sessions, I have also received questions about what kind of behavior I am looking for in the DEM simulations since I only mention that the "Discrete Particle Model simulates behavior". This is mostly because while I am working for Multi-Scale Mechanics Group in the Faculty of Engineering Technology, my primary audiences (and myself) have a background in Chemical Engineering. Therefore, when my supervisors mentioned that this project is partly about simulation of Granular materials, I immediately understood it was mechanical behavior. However, my peers brought up a good point if I am also looking for particle charging behavior or thermal behavior. In the later version of the presentation and this final research proposal, I have included what me and my supervisors are looking for. 
\par
After submitting my first Introduction, I got a review from Hanneke about the citation styles I am using. At that time, I was very new to LaTeX and did not know which citation styles that could match the requirements of the information on each cite, so I ended up going with a default numerical style. After the introduction presentation, I messaged another peer who has also started using LaTeX. She was happy to provide me with the correct bibliography style, including author, journal, section, and doi for each bibliography. 

I have also received feedback on the Introduction from my supervisors on the timeline of the assignment: Instead of fixing a time to start writing the report (week 5), I should instead start with the writing as soon as possible - since some chapters of the report - about the technologies used, background knowledge, and simulation procedure, for example, can already be written without having simulation results. I agree with his idea and stated in the Timeline section that I will start with the writing early to avoid snowballing in the last weeks. After hearing from the module coordinator, I have also included the timeline section on the progress meeting. Before, I planned to have one checkpoint meeting a week to discuss the progress I had made the last week and what I have in my schedule for the coming week. However, it was an excellent point since scheduling a progress meeting will provide a time to discuss how far I have gone with the project so far and am I possible to finish everything in time. 

Finally, I want to express my gratitude to the people who have helped me to develop the final version of the Research Proposal: Lieke Pieters, Maarten Duikersloot, Lara Jansen, Nick Witmarsum, and Vera Ottens in my peer group, Leonie-Krab Husken, Hanneke Becht on general matters and citations, Professor Wim Brilman, and also to my research group with Thomas Weinhart, Anthony Thornton, and Chen Kuan for academic suggestions. There is probably more small improvements in this Research Proposal that was pointed out by others that have not been mentioned in the rebuttal since  sometimes I fixed it right after I received the feedback. 



\end{document}